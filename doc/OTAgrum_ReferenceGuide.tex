% Copyright (c)  2010-2011  EDF-EADS.
% Permission is granted to copy, distribute and/or modify this document
% under the terms of the GNU Free Documentation License, Version 1.2
% or any later version published by the Free Software Foundation;
% with no Invariant Sections, no Front-Cover Texts, and no Back-Cover
% Texts.  A copy of the license is included in the section entitled "GNU
% Free Documentation License".




%%%%%%%%%%%%%%%%%%%%%%%%%%%%%%%%%%%%%%%%%%%%%%%%%%%%%%%%%%%%%%%%%%%%%%%%%%%%%%%%%%%%%%%%%% 
\section{Reference Guide}

The aGrUM library provides efficient algorithms to create and manipulate graphical models. A particular case of such models is the class of Bayesian Networks (BN), which is of first interest in association with OpenTURNS.

\subsection{Bayesian networks}

The following lines are partially extracted from the Wikipedia Encyclopedia. For an in-deepth presentation of the BN theory and algorithms, the reader could read the references indicated in   section \ref{ref}.\\

A {\itshape Bayesian network}, {\itshape belief network} or {\itshape directed acyclic graphical model} is a probabilistic graphical model that represents a set of random variables and their conditional dependencies via a directed acyclic graph (DAG). In this DAG, edges represent conditional dependencies; nodes which are not connected represent variables which are conditionally independent of each other. Each node is associated with a probability function that takes as input a particular set of values for the node's parent variables and gives the probability of the variable represented by the node. \\

The manipulation of a Baysesian network is called {\itshape inference}. Efficient algorithms exist that perform inference and learning in Bayesian networks. 


\subsection{References and theoretical basis}\label{ref}

\begin{itemize}
  \item[1] {\itshape Reseaux Bay�siens - aGrUM}, PH. Wuillemin, Journ�e Utilisateurs Open TURNS 2010, http://share.openturns.org
  \item[2] to precise 
  \item[3]  Wikipedia, www.wikipedia.org, reserach word 'Bayesian Network'
\end{itemize}